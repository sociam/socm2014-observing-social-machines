% THIS IS SIGPROC-SP.TEX - VERSION 3.1
% WORKS WITH V3.2SP OF ACM_PROC_ARTICLE-SP.CLS
% APRIL 2009
%
% It is an example file showing how to use the 'acm_proc_article-sp.cls' V3.2SP
% LaTeX2e document class file for Conference Proceedings submissions.
% ----------------------------------------------------------------------------------------------------------------
% This .tex file (and associated .cls V3.2SP) *DOES NOT* produce:
%     1) The Permission Statement
%     2) The Conference (location) Info information
%     3) The Copyright Line with ACM data
%     4) Page numbering
% ---------------------------------------------------------------------------------------------------------------
% It is an example which *does* use the .bib file (from which the .bbl file
% is produced).
% REMEMBER HOWEVER: After having produced the .bbl file,
% and prior to final submission,
% you need to 'insert' your .bbl file into your source .tex file so as to provide
% ONE 'self-contained' source file.
%
% Questions regarding SIGS should be sent to
% Adrienne Griscti ---> griscti@acm.org
%
% Questions/suggestions regarding the guidelines, .tex and .cls files, etc. to
% Gerald Murray ---> murray@hq.acm.org
%
% For tracking purposes - this is V3.1SP - APRIL 2009

\documentclass{acm_proc_article-sp} 

\usepackage{hyperref}
\usepackage{graphicx}

\begin{document}

\title{7 Billion Home Telescopes: Observing Social Machines through Personal Data Stores}
% \subtitle{[Extended Abstract]
% \titlenote{A full version of this paper is available as
% \textit{Author's Guide to Preparing ACM SIG Proceedings Using
% \LaTeX$2_\epsilon$\ and BibTeX} at
% \texttt{www.acm.org/eaddress.htm}}}
%
% You need the command \numberofauthors to handle the 'placement
% and alignment' of the authors beneath the title.
%
% For aesthetic reasons, we recommend 'three authors at a time'
% i.e. three 'name/affiliation blocks' be placed beneath the title.
%
% NOTE: You are NOT restricted in how many 'rows' of
% "name/affiliations" may appear. We just ask that you restrict
% the number of 'columns' to three.
%
% Because of the available 'opening page real-estate'
% we ask you to refrain from putting more than six authors
% (two rows with three columns) beneath the article title.
% More than six makes the first-page appear very cluttered indeed.
%
% Use the \alignauthor commands to handle the names
% and affiliations for an 'aesthetic maximum' of six authors.
% Add names, affiliations, addresses for
% the seventh etc. author(s) as the argument for the
% \additionalauthors command.
% These 'additional authors' will be output/set for you
% without further effort on your part as the last section in
% the body of your article BEFORE References or any Appendices.

\numberofauthors{6} % in this sample file, there are a *total*
% of EIGHT authors. SIX appear on the 'first-page' (for formatting
% reasons) and the remaining two appear in the \additionalauthors section.
%
\author{
% You can go ahead and credit any number of authors here,
% e.g. one 'row of three' or two rows (consisting of one row of three
% and a second row of one, two or three).
%
% The command \alignauthor (no curly braces needed) should
% precede each author name, affiliation/snail-mail address and
% e-mail address. Additionally, tag each line of
% affiliation/address with \affaddr, and tag the
% e-mail address with \email.
%
% 1st. author
     \alignauthor Max Van Kleek\\
     \affaddr{Web and Internet Science}\\
     \affaddr{University of Southampton}\\
     \affaddr{Southampton, UK}\\
     \email{emax@ecs.soton.ac.uk}
% 2nd. author
     \alignauthor Daniel Alexander Smith\\
     \affaddr{Web and Internet Science}\\
     \affaddr{University of Southampton}\\
     \affaddr{Southampton, UK}\\
     \email{ds@ecs.soton.ac.uk}
%
     \alignauthor Ramine Tinati\\
     \affaddr{Web and Internet Science}\\
     \affaddr{University of Southampton}\\
     \affaddr{Southampton, UK}\\
     \email{rt506@ecs.soton.ac.uk}
     %
     \and 
     %
     \alignauthor Kieron O'Hara\\
     \affaddr{Web and Internet Science}\\
     \affaddr{University of Southampton}\\
     \affaddr{Southampton, UK}\\
     \email{kmo@ecs.soton.ac.uk}
%
     \alignauthor Wendy Hall\\
     \affaddr{Web and Internet Science}\\
     \affaddr{University of Southampton}\\
     \affaddr{Southampton, UK}\\
     \email{wh@ecs.soton.ac.uk}
%
     \alignauthor Nigel Shadbolt\\
     \affaddr{Web and Internet Science}\\
     \affaddr{University of Southampton}\\
     \affaddr{Southampton, UK}\\
     \email{nrs@ecs.soton.ac.uk}
}
% There's nothing stopping you putting the seventh, eighth, etc.
% author on the opening page (as the 'third row') but we ask,
% for aesthetic reasons that you place these 'additional authors'
% in the \additional authors block, viz.

% \additionalauthors{Additional authors: John Smith (The Th{\o}rv{\"a}ld Group, email: {\texttt{jsmith@affiliation.org}}) and Julius P.~Kumquat (The Kumquat Consortium, email: {\texttt{jpkumquat@consortium.net}}).} \date{30 July 1999}

% Just remember to make sure that the TOTAL number of authors
% is the number that will appear on the first page PLUS the
% number that will appear in the \additionalauthors section.

\maketitle
\begin{abstract}
Web Observatories aim to develop techniques and methods to allow researchers to interrogate and answer questions about society through the multitudes of digital traces people now create.  In this paper, we propose that a possible path towards surmounting the inevitable obstacle of personal privacy towards such a goal, is to keep data with individuals, under their own control, while enabling them to participate in Web Observatory-style analyses \emph{in situ}.  We discuss the kinds of applications such a global, distributed, linked network of Personal Web Observatories might have, a few of the many challenges that must be resolved towards realising such an architecture in practice, and finally, our work towards a fundamental reference building block of such a network.
\end{abstract}

% A category with the (minimum) three required fields
\category{H.m}{Information Systems}{Miscellaneous}

\keywords{Web Observatories, Personal Data Stores, distributed systems, personal information environments} % NOT required for Proceedings

\section{Introduction}

The concept of a Web Observatory~\cite{Tiropanis2013,Hall2014} was introduced to investigate methods and mechanisms by which people, as a collective society, could be effectively studied in academic research settings, through the archival and analysis of the information traces they created online. As such traces have become increasingly rich, driven by both increased use of the Web and the onslaught of always-on smartphones, wearable sensors, and other devices that can measure the activities people perform on and off- the Web, two shifts have occurred. The first is that the boundaries between the activities previously considered ``off-line'' and those that were considered ``online'' is rapidly dissolving, meaning that all activities are being increasingly reflected in information about them on-line. A result of this is that the quantity, fidelity, sensitivity, and resulting value of this information is increasing - both in terms of potential value to individuals (as a multipurposable accurate record of their activities), and to third parties seeking to offer and provide services to people based on their lifestyle(s) and needs. 

An implication of these two trends is that Web Observatories will no longer be solely about Web or what we currently think of as ``Web-based activities'' such as participating in online communities, social networks, and so on; rather, these observatories will be about the individual, multifaceted lives of people. From this perspective, it is unsurprising that significant privacy concerns may be raised about the large-scale collection of such data, whether they be for scientific study or commercial application. For example, even efforts driven by public bodies such as the NHS, such as the newly founded \emph{Care.data}~\footnote{Care.data - A Modern Data Service for the NHS \url{http://care.data}} have received widespread criticism (e.g. \cite{RameshNHS}) about its aggregation of millions of Britons' anonymised NHS patient records, even though such collection could drive medical research that might greatly advance the collective wellbeing \cite{de2006use}.

This in general is reflective of a core dilemma faced in the building of such observatories; Web observatories will contain information of increasing potential value to making fundamental advances across research domains (spanning medicine, to human-centred design, to cultural anthropology, for example) but such repositories also represent unprecedented privacy risks and targets for identity thieves, misuse by commercial entities and so on, being comprised of aggregations of detailed, high-fidelity information about people's lives. 

In this position paper, we examine one potential solution path towards resolving this dilemma: a technical architecture that changes the core assumptions surrounding the roles of data observer, aggregator, and broker proposed in Web Observatory research thus far. Specifically, we introduce the notion of \emph{personal data store} as a core atomic component in a new kind of Web observatory; one that is purely distributed and in the collective control of all of its data sources -- the individuals whose data form the observatory.  We approach this idea by outlining the key functions of a Web Observatory, through what they are meant to achieve, which we follow with a definition of Personal Data Stores, including a summary of work done in this space before now. Then, we follow this up with the technical and societal implications of applying PDS architectures to building Web Observatories, focusing on identifying core challenges in this space, including preserving anonymity and privacy of members while promoting data sharing in such settings.

% The Web has profoundly changed the ways that individuals live their day-to-day lives
% through the connectedness afforded by the ability to connect with anyone, anywhere
% and anytime and the massive quantiites of information repositories individuals
% have created for eachg others' use.

% information traces they generate on-line by which society can be 
% systematically studied through digital information traces created from the
% interactions of individuals on the Web \cite{wobs}

\section{What are Web Observatories?}
A Web Observatory is a platform consisting of both a technical architecture and governance to enable the collection, sharing, querying, and analysis of Web Data \cite{Hall2013, Tiropanis2013}. Given that the Web is a rich resource of the current state of the world, the aim of such Web observatories was set to provide a means to monitor, analyse and understand the activity of humans, both as individuals and as a collective. To do so, a core capability of such observatories is to combine information from many disparate streams of data generated by independent Web-based sources, spanning services, social network platforms, applications and so forth, into integrated coherent data models.
 
\section{What are Personal Data Stores?}

The rise of ``Web 2.0'' was marked by transition of the Web from an information publishing medium to being a general platform for all sorts of human interaction, spanning from synchronous point-to-point interaction to many sorts of one-to-many information exchange mechanisms. As Web platforms became more sophisticated and complex, however, we also observed a trend towards greater centralisation; although many factors were involved, among them was the fact that building complex Web services and applications simply required more investment and expertise than most individuals could themselves muster; therefore, the construction of such services quickly became the domain of venture-backed startups. These startups, the nascent Facebooks, Dropboxes and Googles quickly amassed huge quantities of personal information as individuals flocked to their use for their services and capabilities. Seeking to derive revenue from such troves of user information, such companies forged the first versions of now a multi-billion pound a year surveillance-and-analytics business model. Although the kinds of content being amassed began as a few social network profiles and blog posts, it quickly grew to encompass the entirety of personal data people keep {\em or generate}, from files and documents to film and music archives. 

Thus began a migration of personal digital artefacts from individually-administered personal computers into various information spaces of the Web. The aim of PDSes is to start to re-balance the this data inequality by bolstering the capabilities of individuals for managing, curating, sharing and using data themselves and for their own benefit. The idea is not for such capabilities to replace services, nor for individuals to take their data out of the rich ecosystems that exist today (a feat which would be practically impossible, not to mention potentially destructive), but instead to enable people to collect, maintain and effectively derive value from their own data collections directly on the device(s) under their control. The combination of such capabilities and derived value provides an incentive for individuals to take responsibility for, and invest effort in, the preservation and curation of their data collections, turning to external third parties for specialised services only where needed. The aim of such development would be to try to restore some balance by providing a locus for subject-centric management of data, to complement (and in some cases replace) the current paradigm of organisation-centric data management.

Arriving at an operational definition, we define PDSes as follows:

\begin{quote}

	A personal data store is a set of capabilities built into a software platform or service that allows an individual to manage and maintain his or her digital information, artefacts and assets, longitudinally and self-sufficiently, so it may be used practically when and where it can for the individual's benefit as perceived by the individual, and shared with others directly, without relying on external third parties. 

\end{quote}

This description leaves undefined the kinds of activities that might constitute ``managing'', ``maintaining'', ``controlling fully'' or ``using'' this information, nor even what kind(s) of information, owned by whom, that we are talking about. Fortunately, significant insight pertaining to many ways individuals store, archive and retrieve information, in both on-line and off-line contexts throughout the course of many every-day activities, has been the focus of a studies of the field of Personal Information Management (PIM) (e.g., \cite{sellen2003myth,bernstein2008information,van2011finders}).  Such studies have documented the breadth and often idiosyncratic nature of personal information practices, driven by both the fragmented nature of people's information spaces (arising, in part from the lack of integration among apps and siloed data sources of the Web), and the remarkable ingenuity with which individuals often worked around such limitations in order to manage their information archives.  Since PIM studies have uncovered, in nearly equal parts, areas where digital information tools have served people well, and those where they have fallen fantastically short, this literature served as a convenient starting point for deriving the needs for PDSes.  Our design process for our PDS, described later, thus started with a broad consolidation of results of these studies~\cite{van2012decentralized}.  

\section{Personal Web Observatories}

Combining the two ideas of a PDS with the goals of a Web Observatory, a logical first step we propose is that of a \emph{Personal Web Observatory} (PWO), a technical platform that, first and foremost, enables individuals to consolidate and archive their data currently dispersed among multiple sources. Then to use such a consolidated archive to serve as a kind of ``analytical mirror'' that can enable an individual to accurately gauge and reflect upon the multifacted states of their lives and wellbeing. Such consolidated data could be used, for example, for better time budgeting, stress management, budget-planning (through the consolidation of data streams representing spending), fitness and health management (such as through sensed data streams representing the individual's vital statistics and activities). 

With more sophisticated functionality, such a PWO might monitor one's social interactions, and correlate such interactions with states of wellbeing; do certain people seem to be the sources of stress or enjoyment? Similarly, such information might be used to `debug' an individual's other states of wellbeing, such as to identify why random, sporadic headaches might be occurring, such as by correlating such incidences with particular activities, sleep levels, presence in particular locations, with certain times of the year or periods of the month, or with certain activities. Such ``small data'' analytics, while sparse, could be made statistically viable when gathered longitudinally over time, and offer the advantage that they reflect a single person's idiosyncratic patterns and correlations.

\section{Linked, Distributed, Person Web Observatories}

A next logical step from a PWO, then, is towards overcoming the single-individual limitations of a PWO by enabling individuals to combine their data.  The remainder of this paper examines what such a capability would entail, and proposes our progress towards a potential implementation of such a system, a platform we call INDX. 

\subsection{Scenarios}

Prior to identifying barriers to achieving such a goal, we first identify the kinds of \emph{usage scenarios} we envision linking PWOs might enable.  

\subsubsection{Distributed QA}

One broad class of uses can be thought of as a mixture of collaborative software and online Question-Answer sites~\cite{harper2009facts}, in which individuals can issue distributed queryies to a community for things that he/she needs to know.  Like the now-defunct distributed QA service Aardvark~\cite{horowitz2010anatomy}, such queries might be cast to a specific set of people (such as acquaintances or members of an organisation, for example), or, they might be routed to the most qualified or available individuals.  Unlike Aardvark, however, in which users answered all questions, in the PWO scenario, such queries might be posed in a machine-parseable form to allow individuals' PWOs to automatically service them.

Perhaps the most beneficial capability might come from the ability to aggregate responses to such queries automatically across an entire population.  Such a capability could be used to allow individuals to gather large scale statistics useful for computing metrics such as bounds for realising differential privacy \cite{dwork2006differential} policies.  For example, it could be possible to ask specific demographic questions such as \emph{``How many people live in Southampton who have a pet terrier, wear glasses and work at a Costa?''}.  While seemingly oddly specific, the response to such a query might be used to determine the degree to which an individual who wishes to remain anonymous in public (to protect themselves, for example) might choose to be more selective about what they disclosed about their activities or employment.

However, for such scenarios to be realised, methods to ensure that such questions can be answered themselves without violating the privacy of the responders, question answers, or intermediaries will likely be essential. 

\subsubsection{Publishing Profiles for Data Analysis}

While the previous scenario discussed a ``pull'' approach to distributed data analysis, another approach is to essentially allow groups of individuals to ``push'', or expose, ``public profiles'' of particular aspects of themselves for purposes that serve the collective good.  For example, if smart automobiles in the future volunteered their coordinates (in a privacy preserving way) in real time to a collective tally of road congestion, people could in real time determine which routes to avoid, entirely obviating the need for a centralised service to do so (such as the Waze App\cite{blatt2013technological}). 

Just like in the previous scenario, an essential privacy requirement might be for such profiles to provide selective but authoritative statements about someone or something being in a particular state or having a particular property, without identifying the individual or thing that has it.  Similarly, it would have to be guaranteed that multiple such profiles could not be attributed ot the same source, a well-known form of \emph{disclosure intersection attack} \cite{danezis2005statistical}.

\subsubsection{Ethnographic Enquiry and Web Science}

Beyond the specific sorts of data push and pull to support the kinds of queries and analysis described above, a third set of applications might be in supporting effective ethnographic enquiry and analysis in such environments, were individuals possess vast repositories of information about their daily activities and experiences.  Answering such a question may necessarily involve addressing the issue of information legacy, and how one might support the effective preservation, ownership rights and control of life activity databases stored in people's PWO from one individual to the next, after they have died.  The complex moral and culture-specific issues pertaining to addressing such problems have been discussed extensively elsewhere (e.g. \cite{shields1996cultures,fernandez2001moral}) and are particularly salient for PWOs, where individuals might be in possession of complete records of their own life histories.

\subsection{Towards Linked PWOs: Challenges}

% The genesis of an individual-centric personal data archive pre-dates digital computers entirely, to Vannevar Bush's Memex vision of 1945\cite{memex}, which proposed a mechanical framework for supporting the collection, archiving, and organisation of information to facilitate later cross-reference and retrieval. Among the important contributions of this article was the significant emphasis on reducing the effort needed to capture and retrieve information, due effort being the primary impediment towards effective and frequent information use. To this end, Memex proposed that individuals could wear capture devices on their bodies (a camera strapped to the forehead), store such information compactly, conveniently and indefinitely, and retrieve it later through an associative mechanism modelled upon the human memory, queried naturally via gesture.

% Two additional early projects that explored how such information archives might be realised were Ted Nelson's Xanadu \cite{nelson1987literary} and Douglas Engelbart's NLS \cite{engelbart1968research}. Both proposed that information environments could be interlinked through a global network of knowledge sharing, demonstrating many ideas in the 1960's that would not be realised in commercial systems for decades. While the former focused on hypertext and distributed collaboration, the latter focused on structured data collections, including data navigation, creation and management. Engelbart demonstrated an actual prototype of NLS in 1969, capable of synchronous collaboration, complete through a graphical user interface, that incorporated dynamic hierarchies, hyperlinks, and multi-view representations 

% The introduction of the personal computer (PC) in 1984 provoked the development of the first generation of digital personal information management tools, consisting of a variety of application software products designed to help individuals create and maintain collections of digital data, ranging from flexible, schema-agnostic personal database systems like Filemaker \footnote{Filemaker - \url{www.filemaker.com}}, to specific data types, such as digital calendaring tools, and ``digital Rolodex'' address books. Seeking to appeal to the first generation of personal computer users, many of these applications borrowed metaphors from paper-based information collection tools, from the notion of ``documents'', to that of files and folders, and even notebook ledgers and personal diaries. Along with this deliberate shaping of digital information into forms designed to be familiar with paper information organiser came interaction metaphors and organisation methods for them; from deletion of information by ``throwing in the rubbish bin'' to ``desktop'' and ``filing cabinet''-based based information organisation and arrangement.

% Meanwhile, research in personal information management continued to pursue the vision put forth by Memex, towards methods of automatically building archives of personal life activities and experiences, so that these might be used as external memory prostheses. The pursuit of this vision was partially responsible for the development of handheld and early wearable computing technology, such as the Xerox PARC Tab \cite{schilit1993parctab}, arguably the first hand-held computer, which ran arguably the first automatic location-based personal lifelog, PEPYS~\cite{newman1991pepys}. Many systems that captured other aspects of context and activities soon followed, such as the Remembrance Agent by Rhodes et al., and the life archive by Clarkson et al., both at the MIT Media Lab's ``Cyborg'' Wearable Computing group. Since the breadth of kinds of activities and experiences that such systems captured transcended paper documents, such research required re-thinking the shape of data away from paper-metaphors to other kinds of collections, including \emph{information streams} (e.g., Lifestreams \cite{fertig1996lifestreams}) and chronological \emph{lifelogs}, such as MyLifeBits \cite{gemmell2002mylifebits}.

% % Good selection of stuff in the previous para - but why comment out the next sentence? Still seems relevant, although we should probably take out the “next decade” reference

% % The next decade saw specific evaluations of lifelogging in various specialised contexts, including healthcare for chronic disease maintenance, including memory prosthesis applications for Alzheimer's patients \cite{}, and cognitive behavioural therapy.

% The third, and potentially most profound, transformation of digital information tools occurred with Web 2.0, the rise of a ``social Web'' replete with dedicated apps and services for managing and sharing nearly any kind of previously imagined personal information, ranging from the sensitive and intimate to the public. 

% Meanwhile, the data proliferated too. Seeking to monetise the flood of information people were putting online, markets for personal information quickly began to emerge, prompting concerns over privacy, security, and rights of access, which in turn have driven government and regulators' interest towards giving citizens more protection over various aspects of how data about them could be collected and handled. This led to international efforts to craft data protection legislation, as discussed above. In terms of the provision of data to individuals, such legislation so far has focused on allowing data subjects to inspect the data an organisation holds about them; on receiving a subject access request, the organisation is obliged to correct inaccuracies, and to respect requirements that the data is not used in any way which may cause damage or distress, and that the data is not used for direct marketing purposes.

% 
% \subsection{early PDS concepts}

% However, this is a fairly minimal power which is hardly congruent with the increasing clamour concerning rights to data, including the spread of enforced transparency of data from the private sector \cite{fung2007} and the vogue for freedom of public sector information \cite{ohara2014}, and technology (and technology policy) together with new attitudes to transparency bring more possibilities. In the UK, a government initiative called \emph{midata} \cite{midata} is working to bring about the logical next step of customers getting direct and unfettered access to data kept about them by companies (other similar initiatives include the US Blue Button initiative\footnote{\url{www4.va.gov/bluebutton/}.} and the French Mesinfos group\footnote{\url{mesinfos.fing.org/}.}). The ultimate success of \emph{midata} will be contingent on several important steps in both technology and regulation, most particularly including realising effective tools such as personal data stores for letting individual users easily consume, consolidate and make use of this data once it is made available.

% Independent of such legislative approaches, both academic and industry-led efforts also began to commit resources to research towards identifying ways that end-user citizens might, in the face of the vast growing repositories of data being held about them, enjoy more control and privacy. An academic consortium known as \emph{Vendor Relationship Management} (VRM) at Harvard's Berkman Center was realised to conduct multifaceted research into socio-legal-econo-technical approaches that might be employed. Among the products of this research was a vision that users might stand as their own information brokers, and start to act as peers with service providers, capable of negotiating fair and equitable mutual terms of data use during interactions with them\cite{agustin2001vendor}. Out of this work emerged the earliest mentions of Personal Data Stores for realising such capabilities in the context of online e-commerce, inspiring more than a dozen different Personal Data Store offerings, platforms and services backed by commercial start-ups since 2001~\cite{ctrlshift}. 

% As an example, consider Mydex, whose proof-of-concept offering dates back to 2009 \cite{heath2013}. Mydex designers worked with data-handling organisations to develop systems to support data transfer and sharing governed by consent and identity verification. Design principles included putting the individual PDS owner in sole charge of consent giving and revocation with a simple `on/off' switch; giving the individual sole access to the private encryption key; verification of all organisations wishing access to data; and comprehensive data sharing agreements going beyond Data Protection Act protections. The business model for Mydex is still experimental, but currently the idea is to fund the stores by charging organisations for access to data; if the charge is set low enough, then they should save by side-stepping other access costs (e.g. the costs of writing a letter to the data subject). The Mydex services are currently free of charge to the individual. Mydex exploits cloud infrastructure with open source software, but its PDSes are discrete collections of files encrypted and controlled by the individual, including --- and this seems prescient after the Snowden revelations\footnote{\url{www.theguardian.com/world/the-nsa-files, www.ub.uio.no/fag/informatikk-matematikk/informatikk/faglig/bibliografier/no21984.html}.} --- the ability to choose the location of the data centre in which the PDS is stored. Similar open source personal data storage containers include The Locker Project\footnote{\url{lockerproject.org}.}, data.fm\footnote{\url{data.fm}.}, Owncloud\footnote{\url{owncloud.org}.}, and OpenStack\footnote{\url{www.openstack.org}.}, each of which provides various degrees of easy-to-set-up `personal cloud' software that can be used to store and host content on the user's own server on the Web.

% %The potential impact of personal data store technology towards driving new models of e-commerce and new experiences for end-users has been the focus of substantial interest recently among independent research organisations. 
% A consistent theme of commentary in this area has seen Personal Data Stores (PDS) as important, if not essential, capability for end-users towards growing a healthier global ``personal data ecosystem''. For example, an independent study commissioned by The World Economic Forum documented ways that the value of personal data might be further ``unlocked'', citing Personal Data Stores as a core enabling mechanism to turn end-users from consumers into more autonomous data brokers\cite{WEF-report}. A separate comprehensive analysis by \emph{Ctrl-Shift} on emerging commercial PDS platforms and offerings projected an enormous economic opportunity for PDS services in the next five years\cite{ctrlshift}. In their view, PDSes are the key to making sense of the myriad data sources that now surround us, from data we volunteer, to the data that commemorates observations of our behaviour, to the data inferred about us, combined with the data we generate via management of our personal affairs (e.g. in health or finance), and also bringing in data about our activities as customers or consumers, including our contributions to loyalty card schemes.

% \subsection{Failure to Launch: Barriers to PDS Adoption}

% Yet despite the extensive needs analysis and market potential identified, early personal data store offerings have thus far failed to attract substantial attention from users. While a number of factors are likely responsible, so the lack of interest among users has been attributed to the fact that many of initial PDS platforms have sought to simply re-create existing end-user experiences offered by popular apps and Web platforms, rather than creating new functionality. Despite the benefit that these PDS offerings provide in terms of data security, users are often less compelled to try something new if the tangible experience nothing new, while data security remains an abstract, inestimable threat which does not necessarily easily compel behaviour change \cite{bandura1977self}. Finally, since the very purpose of PDS offerings is to protect user data from third party access, these platforms cannot derive revenue from user data and must resort to subscription models --- always less attractive to new users than than offerings that are completely free to use. 

% On top of these suppressors of the positive impulse to manage data, we must also remember that the markets work pretty well for some (the most powerful) operators, and so there is a great deal of inertia around. A dogmatic view of revealed preferences of course suggests that individuals' lack of interest in the technology shows they have no desire to curate their own data. They happily click on privacy policies they have never read, and they buy the goods that are marketed to them, at least in sufficient quantities to justify the marketers' costs. `Push' models seem to be in the ascendant, because the data oligarchs are the only agents with access to the bigger picture of what data is held about you, what can be inferred from that data, what services are available, and how you relate to the general data context. `Pull' models struggle, because individuals cannot see the opportunities that are around. In short, the argument is often made that the technological direction of travel is more or less set, that it serves the public good, that the public is uninterested in any alternative, and so, to coin a phrase, ``get over it.'' This deterministic model has been called Zuckerbollocks \cite{ohara2013}, and it is important to challenge and resist it.

% % eMax: love it! and this next paragraph is incredible

% Heath et al write \cite{heath2013} that ``there is market evidence that [the person-centric model of control over personal data] is starting to establish itself,'' but even they see a challenge to getting the model to work. Three conditions need to obtain simultaneously, on the account of Heath et al: PDSes must (i) make life simpler/better for the individual, (ii) appeal to data consumers by solving some of their problems (e.g. costs, or legal liability), and (iii) solve some pressing challenge that is holding back developers and entrepreneurs in this space. To these three, we can add a fourth, which is to re-jig current data protection thinking. At the moment (2014), there are three key roles in the standard model of data protection: the data subject, the data controller and the data processor. The owner of a PDS is none of these (or none exclusively --- he or she is likely to be all three at various times), and it is hard to see how individuals can exercise autonomous control over the data that affects them without some recognition of them as active agents in a different kind of role. Furthermore, data protection legislation is intended to cover cases of personal data being misused by others; it does not cover cases where individuals accidentally (or deliberately) identify themselves. Of course, this is a reasonable starting point for protection, but if it is the only principle, it means that if an individual `takes charge' of his or her data, he or she {\em loses} the cover of Data Protection Acts.

% %This context thus has started to highlight some of the difficulties with introducing new PDS approaches within the existing, highly competitive and lucrative Web service and platform environment which has been powered by the economy of personal data. This drove us seek 	


The goal of realising the previous visions of interlinked PWOs requires addressing a large host of challenges, from those pertaining to the PDS-level challenges of longitudinal information keeping, the many privacy-related challenges pertaining to effective information disclosure without privacy loss, dealing with attackers and identity thiefs.  We outline a few such challenges we have not already discussed, below.

\subsubsection{Long-term Data Maintenance}

Enabling individuals to keep their data safely for a long time, while ensuring its continued accessibility and usefulness impacts both the data formats and methods used to store them.  For example, since a person's physical computational hardware is likely to fail with age, methods need to be in place for ensuring robustness to such failures, such as multi-device replication and easy migration from older to new devices over time.   Moreover, as evidenced by Moore's law \cite{schaller1997moore}, since the technical capabilities and properties of such data storage devices and platforms are likely to change fundamentally, PWOs must be designed to accommodate (and take advantage of) such changes as they arise. 

\subsubsection{The End-user Expertise Gap}

A core philosophy of participant-centric PWOs is that the user assumes all responsibilities for managing and securing their data, as well as making critical decisions regarding their privacy, their own security, and ways to apply PWOs to their tasks and responsibilities.  This saddles users with significant burdens which may both be extremely effort-expensive, but that individuals might actually have no expertise, experience or interest in doing.  From this perspective, it is no surprise that, even in these comparatively simple days of ``Web 2.0'' data management services, individuals have been motivated to outsource maintenance of their data to third parties, such as cloud providers. 

%In order to facilitate autonomy from such services, therefore, PDSes must seek to support directly, and automate where possible, tedious data maintenance tasks that have plagued PC users for decades. Such automation could both ensure compliance for promoting data security and integrity, such as continuous backup regimes, thereby countering recent studies of the extremely low compliance of personal data backup and security maintenance practices \cite{chervenak1998protecting, grasso2006survey}.

%A separate set of challenges arises from the shift back from service-provider controlled data storage to a user-centered model of data management. Although this will re-empower users to control the organisation of their data spaces, and eliminate the pervasive problem of data fragmentation \cite{karger2006data}, \cite{heath2011linked}, the challenge with the increased flexibility that this approach affords is that it requires re-consideration of how third-party applications and services can interact with such data, which have traditionally been pre-defined to operate on a fixed, typically application-provider established, set of data representation(s) and manipulations. In a consolidated, user-centric data model, on the other hand, such representations may be be specified or modified by the individual, or by some other third-party application(s) on behalf of them, and thus applications themselves must be designed to accommodate such variability among representations.

% Data Security and Legal compliance
%The need to comply with local, national and international data handling requirements pose a fourth set of challenges. In particular, if PDSes are to support the storage of identifiable information, or more critically, regulated sensitive information such as individuals' medical records, then PDSes must implement a variety of security standards (e.g. \cite{mccallister2010guide}) to ensure secured storage. Perhaps more difficult might be achieving compliance with the additional data handling requirements imposed by these regulations beyond how it is stored and encrypted; in particular, key handling requirements and guaranteeing aspects of physical access to the machine(s). The integrity of data must also be secured --- for instance, although a patient should have the right to challenge and correct inaccurate medical data, if the PDS is to store a version of medical data that is likely to be used (for example, in support of medical treatment in a foreign country), the data would need not only to be accurate, but also of appropriate provenance in order to be properly adapted to the standard workflows of medical treatment.

\subsubsection{Third-party Interoperability}

A separate set of challenges arises from the shift back from service-provider controlled data storage to a user-centred model of data management. Although this will re-empower users to control the organisation of their data spaces, and eliminate the pervasive problem of data fragmentation \cite{karger2006data}, \cite{heath2011linked}, the challenge with the increased flexibility that this approach affords is that it requires re-consideration of how third-party applications and services can interact with such data, which have traditionally been pre-defined to operate on a fixed, typically application-provider established, set of data representation(s) and manipulations.  In a consolidated, user-centric data model, on the other hand, such representations may be be specified or modified by the individual, or by some other third-party application(s) on behalf of them, and thus applications themselves must be designed to accommodate such variability among representations.

\subsubsection{Handling Identifiable Information}

When multiple individuals' PWOs interact and exchange data, the handling of others' data may constitute the handling and storage of \emph{identifiable information}\cite{narayanan2010myths}. The handling of third-party identifiable information places, under many current forms of legislation, in a category which requires them to comply with local, national and international data handling requirements. Such requirements are more sever if some of the data exchanged fall in the category of particular kinds of sensitive information, such as individuals' medical records or histories, in which case PWOs must comply with a variety of stringent requirements (e.g. \cite{banisar1999global}) to ensure secured storage.  

% \subsubsection{Interacting with Third Parties}
% % Protecting Privacy
% Even if individuals use PDSes and PWOs, it is likely that third-party service providers would inevitably continue to profile and amass information about them, as long as it aligned with their incentives to do so (and it is hard to imagine that it will not --- for instance, a service provider may need to gather a large amount of personal data in order to ensure correct and appropriate billing for its services). Thus, if PDSes are to give users the degree of autonomy and independence from profiling, they would need to include privacy-enhancing technologies, such as IP anonymisers, user-agent randomisation and cookie blocking. This may be difficult or impossible to do on ``closed'' platforms such as iOS that prevent these techniques because they are perceived as ``hacking''.

%Perhaps the ultimate set of challenges, however, pertain to accommodating change as it affects both the information itself and the practices and activities surrounding it, over the years that a PDSes is intended to operate. Technologies that bring in new ways that data is used and generated seem to be introduced every quarter, placing new demands how this information needs to be accessed, created and used. The most recent examples include wearable computing and ``always on'' wearable sensor technology, from simple devices such as Fitbits \footnote{Fitbits - \url{www.fitbit.com}} and Fuelbands\footnote{Nike+ Fuelband - \url{www.nike.com/fuelband}} that unobtrusively but nearly constantly measure simple aspects of an individual's activity, to complex computational devices that can both deliver and capture information in high fidelity and quantity anywhere, such as Google Glass\footnote{Google Glass - \url{www.google.com/glass}}. Such devices, as well as innovative new apps in can in some cases bring about changes in norms pertaining to people's activities, including the ways people think about technologies themselves.

\subsubsection{Anticipating Future Needs}
Perhaps the ultimate set of challenges, however, pertain to accommodating change as it affects both the information itself and the practices and activities surrounding it, over the years that a PWO is intended to operate.  Technologies that bring in new ways that data is used and generated seem to be introduced every quarter, placing new demands how this information needs to be accessed, created and used.  The most recent examples include wearable computing and ``always on'' wearable sensor technology, from simple devices such as Fitbits \footnote{Fitbits - \url{www.fitbit.com}} and Fuelbands\footnote{Nike+ Fuelband - \url{www.nike.com/fuelband}} that unobtrusively but nearly constantly measure simple aspects of an individual's activity, to complex computational devices that can both deliver and capture information in high fidelity and quantity anywhere, such as Google Glass\footnote{Google Glass - \url{www.google.com/glass}}.  Such devices, as well as innovative new apps in can in some cases bring about changes in norms pertaining to people's activities, including the ways people think about technologies themselves.

%% Maybe we should move the legal challenges down here??
% Maybe - I've put something about durability in an earlier challenge, but I think that this is separate from the idea of longevity and durability. But the advantage of ending with this challenge is that it is in a sense the biggest, so I like it as it stands. We do need to make sure that this challenge is clearly distinct from the first two. I think it is, but maybe there's a way of phrasing this that makes it clearer.

Looking forward at some of the ways such technologies might impact information activities, some have looked at the possible consequences and implications that ever-increasing information capture and access might have on the kinds of activities mentioned above. While Bell and Gemmel have argued \cite{bell2010total} that such increased capture and access could create near-perfect records of our daily lives, allowing people to examine with unprecedented scrutiny their everyday activities, others such as Mayer-Schonberger have argued that such a utopian views overlooks a great number of potential unintended consequences \cite{mayer-schonberger2013}.  

% Looking forward at some of the ways such technologies might impact information activities, some have looked at the possible consequences and implications that ever-increasing information capture and access might have on the kinds of activities mentioned above.  While Bell and Gemmel have argued \cite{bell2010total} that such increased capture and access could create near-perfect records of our daily lives, allowing people to examine with unprecedented scrutiny their everyday activities, others such as Mayer-Schonberger have argued that such a utopian views overlooks a great number of potential unintended consequences \cite{mayer-schonberger2013}.   

% % Is the above para a direct comment on the preceding challenge? I wonder whether it wouldn't be better placed elsewhere. It's a good point to make - it might even be something for the intro.

The difficulties that this community has encountered have led us to reconsider, from the ground up, the need(s) these platforms are meant to address, so that they can be used to design a platform that will fulfil needs beyond secure data storage, towards new applications that promote the more effective use of data in both personal and social contexts.


% Heath et al write \cite{heath2013} that ``there is market evidence that [the person-centric model of control over personal data] is starting to establish itself,'' but even they see a challenge to getting the model to work. Three conditions need to obtain simultaneously, on the account of Heath et al: PDSs must (i) make life simpler/better for the individual, (ii) appeal to data consumers by solving some of their problems (e.g. costs, or legal liability), and (iii) solve some pressing challenge that is holding back developers and entrepreneurs in this space. To these three, we can add a fourth, which is to rejig current data protection thinking. At the moment (2014), there are three key roles in the standard model of data protection: the data subject, the data controller and the data processor. The owner of a PDS is none of these (or none exclusively --- he or she is likely to be all three at various times), and it is hard to see how individuals can exercise autonomous control over the data that affects them without some recognition of them as active agents in a different kind of role. Furthermore, data protection legislation is intended to cover cases of personal data being misused by others; it does not cover cases where individuals accidentally (or deliberately) identify themselves. Of course, this is a reasonable starting point for protection, but if it is the only principle, it means that if an individual `takes charge' of his or her data, he or she {\em loses} the cover of Data Protection Acts.

%This context thus has started to highlight some of the difficulties with introducing new PDS approaches within the existing, highly competitive and lucrative Web service and platform environment which has been powered by the economy of personal data.  This drove us seek 	

\section{INDX: A Reference PWO ``Atom''}

In this section, we briefly introduce our efforts at designing a reference implementation of first PWO Atom, an open source community platform called INDX\footnote{INDX: A personal data platform - \url{indx.es}}.  While hoping to solve all of the aforementioned problems may seem fool-hearty, the goal of our efforts are to try to identify, rather than solve in an ideal manner, existing methods and technology that can applied to  make incremental progress towards various dimensions of an interlinked, global Personal Web Observatory.  Embracing the philosophy of Richard Feynman (``What I cannot create I do not understand''~\cite{feynman1982simulating}) we have found that the process of designing a PWO itself has surfaced both unanticipated challenges that could be solved with a practical application of (some still emerging) systems architecture best-practices.  We briefly discuss design challenges pertaining to the following four areas: distributed sharing, authentication, synchronisation, longitudinal storage, and anonymous distributed querying.

\subsection{Distributed architecture considerations}

The problems of distributed sharing include issues of trusting external servers to be who they claim; and determining which information should be shared and which should be kept private.  In distributed architectures, exchanging information might  might involve both simple direct point-to-point communications, as well as communications relayed through any number of (potentially untrusted) parties.  In either case, the ability for both communicating parties to establish a secure channel to one another with the guarantee that the other party is the intended one is essential for secure information exchange to be possible.

The problem of authentication is that of being able to verify the identity of entities, including users, within a distributed system. In a traditional system, a user would typically log in to each system explicitly to authenticate with it, typically first establishing a principal for each first.  However, in a distributed system, explicitly establishing principals on every system is inefficient, requiring the creation and maintance of $O(n^2)$ principals.  Distributed identity systems, such as OpenID \cite{recordon2006openid}, WebID\cite{huang2000webid}, or Persona \cite{koshutanski2007distributed}, discussed earlier, meanwhile provide a solution that uses a proof-of-identity mechanism relying on common third parties, which are typically well-known  \emph{distributed identity providers}. We have added OpenID support into the INDX reference implementation to allow users to prove their identities to any other INDX Atom, and support for other protocols is planned for the future.

\emph{Synchronisation} refers to the ability to support concurrent editing of shared information items in a partially-disconnected environment, such as when an INDX node is occasionally powered off or when network connectivity becomes sometimes unavailable.  Allowing shared information items to continue to be edited, even when some nodes where copies are stored are unavailable means that changes must be reconciled when communication among nodes is re-established.  Methods have been devised to support user-intervention-free reconciliation such as \cite{sun1998operational}, and INDX currently takes a simple, opportunistic approach to handle a majority of such cases without user intervention.  Similarly, the challenge of durability against data loss, described earlier, is addressed in INDX through an implementation of the LOCKSS principle~\cite{reich2001lockss}, in which important data is automatically replicated across several INDX instances, located on physically separate and potentially distant locations, to reduce the likelihood of data loss.

Finally, the INDX instances as PWO ``atoms'', for the kinds of applications envisaged earlier, requires consideration of both how such queries can be effectively performed, and ways that individual participants' identities can be effectively protected in the process.  Although this remains an area of active research for INDX, we are drawing upon work in decentralised information indexing and query routing in peer networks (e.g. \cite{cudre2007gridvine,tatarinov2003piazza} as well as methods that preserve anonymity by considering methods such the conceptually simple \emph{k-anonymity} \cite{sweeney2002k} to the theoretically-grounded methods of \emph{differential-privacy}~\cite{dwork2006differential} for protecting participants' privacy under query disclosure. Specifically, we are considering these methods for allowing INDX users to easily express \emph{how} identifiable they want to be, and then automatically deducing an appropriate exposure policy for answering distributed queries.  

% Figure \ref{fig:layercake} illustrates the current core components of the PWO Atom, INDX.

% \begin{figure}
% \centering
% \includegraphics[width=8.5cm]{layercake.png}
% \caption{The INDX layer cake, showing how different technologies interact with each other.}\label{fig:layercake}
% \label{fig:layercake}
% \end{figure}

\subsection{The Wellbeing Observatory}

Our first PWO application for INDX is the Wellbeing Observatory, an application which aims to demonstrate ways that information fusion and distributed query can benefit an individual in a health and wellbeing context. The idea of of the Wellbeing Observatory is to consolidate information from the large number of worn activity sensors and devices that measure individuals' daily activities into a singular, coherent diary of their daily lives.  Abstracting raw sensor signals to approximations of physiological signals also guarantees that, even as sensors are lost, replaced, or made obsolete, the information representation will remain consistent in terms of standard physiological concepts and measurements.

With respect to social PWO functionality, the observatory will offer an `'ask the crowd'' feature which will allow individuals to ask others (with the option of doing so anonymously), which will route the question to an appropriate set of individuals who do not even need to be acquainted.  For example, if an individual is trying to identify the cause of a particular set of symptoms they are experiencing, they might query the crowd for others with similar medical histories, living in their geographic region, or with similar recent activity histories to determine whether others have experienced the same symptoms and why.  An early interface mockup of the functionality we envision in the wellbeing observatory is visible in Figure \ref{fig:pho}.

\begin{figure}
\centering
\includegraphics[width=8.9cm]{indx.png}
\caption{An interface mockup of the Wellbeing Observatory to be integrated into the INDX Personal Data Store platform; this application will showcase sensor data integration into a consolidated representation of an individual, and social querying functionality.}\label{fig:pho}
\end{figure}

\section{Conclusion}

In this position paper, we proposed a technical approach to building a Web Observatories comprised of singular components centred around the individual.  These interconnected components, which may be all different, will be based on the Web and exchange data and interoperate fluidly over time, even as the technologies they are based upon change beneath them.  

Just as the most powerful radio telescopes are formed by thousands of smaller telescopes, arranged and connected in a way to form a coherent array  more capable than any singular node, we feel that a billion-node linked PWO connecting every human's personal data on the planet might one day allow questions about humanity and society to be answered at depths and scales unreachable by any other means or instrument.

\bibliographystyle{abbrv}
\bibliography{socm2014-observing}

\balancecolumns
% That's all folks!

\end{document}
 
